\documentclass{article}

% -------------------- CONFIGURACIÓN GENERAL --------------------
\usepackage[utf8]{inputenc}
\usepackage[T1]{fontenc}
\usepackage[spanish]{babel}
\usepackage{geometry}
\geometry{margin=2.5cm}

\usepackage{amsmath, amssymb, amsfonts} % Matemática
\usepackage{physics}                    % Derivadas, vectores, etc.
\usepackage{siunitx}                    % Unidades (SI)
\usepackage{graphicx}                  % Imágenes
\usepackage{xcolor}                     % Colores
\usepackage{tikz}                       % Gráficos y vectores
\usepackage{pgfplots}                   % Gráficas de funciones
\pgfplotsset{compat=1.18}
\usepackage{float}                      % Posición exacta de figuras
%\usepackage{hyperref}                   % Enlaces internos y externos
\usepackage{fancyhdr}                   % Encabezados personalizados

\AtBeginDocument{\RenewCommandCopy\qty\SI}

% Encabezado y pie de página
\pagestyle{fancy}
\fancyhf{}
\rhead{Luca Di Bene - Física I}
\lhead{Apuntes}
\rfoot{\thepage}

% Numeración de ecuaciones por sección
\numberwithin{equation}{section}

% -------------------- DOCUMENTO --------------------
\title{Ejemplos}
\author{Luca Di Bene}
\date{\today}

\begin{document}

\maketitle
\tableofcontents
\newpage

% -------------------- SECCIÓN EJEMPLOS --------------------
\section{Vectores y Operaciones}

Un vector se puede expresar como:
\[
\vec{v} = v_x \hat{i} + v_y \hat{j}
\]

\subsection{Producto escalar}
\[
\vec{a} \cdot \vec{b} = a_x b_x + a_y b_y
\]

\subsection{Producto vectorial}
\[
\vec{a} \times \vec{b} = \begin{vmatrix}
\hat{i} & \hat{j} & \hat{k} \\
a_x & a_y & a_z \\
b_x & b_y & b_z
\end{vmatrix}
\]

\subsection{Diagrama con vectores (TikZ)}
\begin{figure}[H]
    \centering
    \shorthandoff{>}
\begin{tikzpicture}[scale=1.2]
    \draw[->] (0,0) -- (3,1) node[anchor=south west] {$\vec{A}$};
    \draw[->] (0,0) -- (-1,2) node[anchor=south east] {$\vec{B}$};
    \draw[->, thick, red] (0,0) -- (2,3) node[anchor=south] {$\vec{A}+\vec{B}$};
    \draw[->, dashed, gray] (3,1) -- (2,3);
    \draw[->, dashed, gray] (-1,2) -- (2,3);
    \draw[->] (-3,0) -- (4,0) node[right] {x};
    \draw[->] (0,-1) -- (0,4) node[above] {y};
\end{tikzpicture}
\shorthandon{>}
    \caption{Suma de vectores}
\end{figure}

% -------------------- FUNCIONES Y GRÁFICOS --------------------
\section{Gráficas de funciones}

\begin{figure}[H]
    \centering
    \begin{tikzpicture}
        \begin{axis}[
            axis lines = middle,
            xlabel = $x$, ylabel = $y$,
            grid=both,
            width=10cm,
            height=6cm,
            samples=100
        ]
        \addplot[blue, thick, domain=-5:5]{x^2};
        \addplot[red, dashed, domain=-5:5]{2*x + 1};
        \legend{$y = x^2$, $y = 2x + 1$}
        \end{axis}
    \end{tikzpicture}
    \caption{Gráfico de funciones}
\end{figure}

% -------------------- FÓRMULAS AVANZADAS --------------------
\section{Fórmulas útiles}

\subsection{Cinemática}
\[
x(t) = x_0 + v_0 t + \frac{1}{2} a t^2
\]

\subsection{Energía cinética}
\[
E_k = \frac{1}{2} m v^2
\]

\subsection{Integrales y derivadas}
\[
\int_0^T a(t) \dd{t} = v(T) - v(0)
\qquad
\dv{E}{t} = -P
\]

% -------------------- UNIDADES Y TABLAS --------------------
\section{Unidades y tablas}

\begin{itemize}
    \item Masa: \SI{5.0}{\kilo\gram}
    \item Longitud: \SI{2.0}{\metre}
    \item Tiempo: \SI{1.5}{\second}
\end{itemize}

\begin{table}[H]
    \centering
    \begin{tabular}{|c|c|c|}
        \hline
        Cantidad & Unidad & Símbolo \\
        \hline
        Masa & kilogramo & \si{\kilogram} \\
        Longitud & metro & \si{\metre} \\
        Tiempo & segundo & \si{\second} \\
        \hline
    \end{tabular}
    \caption{Unidades del SI}
\end{table}

% -------------------- REFERENCIAS --------------------
\section{Referencias cruzadas}

La fórmula de energía cinética fue introducida en la sección \ref{sec:energia}, y es clave en mecánica.

\label{sec:energia}

% -------------------- FIN --------------------
\end{document}